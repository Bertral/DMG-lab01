 

% !TeX encoding = UTF-8
\documentclass[11pt,a4paper,twoside,svgnames]{article}

%%%%%%%%%%%%%%%%%%%%%%%%%%%%%%%%%%%%%%%%
%%%%%%%%%%%%%%%%%%%%%%%%%%%%%%%%%%%%%%%%
%
% ENCODAGES, LANGUES, AMS ET AUTRES
%
%%%%%%%%%%%%%%%%%%%%%%%%%%%%%%%%%%%%%%%%
%%%%%%%%%%%%%%%%%%%%%%%%%%%%%%%%%%%%%%%%
%
\usepackage[T1]{fontenc}
\usepackage[utf8]{inputenc}
\usepackage[english,frenchb]{babel}
%
\frenchbsetup{StandardLists=true}
\usepackage{enumitem}
\usepackage{xspace}
\usepackage{amssymb,mathtools,pifont}
\usepackage{xcolor}
\usepackage{graphicx}
\usepackage{listingsutf8}

\lstset{literate=
	{á}{{\'a}}1
	{à}{{\`a}}1
	{é}{{\'e}}1
	{è}{{\`e}}1
	{í}{{\'i}}1
	{ó}{{\'o}}1
	{ü}{{\"u}}1
	{ô}{{\^o}}1
	{ú}{{\'u}}1,
	language=java,
	basicstyle=\footnotesize,
	breaklines=true,
	frame=single,
	showspaces=false,
	showstringspaces=false,
	showtabs=false,
	extendedchars=true
}


%%%%%%%%%%%%%%%%%%%%%%%%%%%%%%%%%%%%%%%%
%%%%%%%%%%%%%%%%%%%%%%%%%%%%%%%%%%%%%%%%
%
% MARGES, ENTÊTES ET PIEDS DE PAGE, TITRE
%
%%%%%%%%%%%%%%%%%%%%%%%%%%%%%%%%%%%%%%%%
%%%%%%%%%%%%%%%%%%%%%%%%%%%%%%%%%%%%%%%%
%
\usepackage[hcentering=true,nomarginpar,textwidth=426.8pt,textheight=650.2pt,headheight=24pt]{geometry}
%
\usepackage{fancyhdr}
\fancypagestyle{plain}{
	\fancyhf{}
	\renewcommand{\headrulewidth}{0pt}
	\renewcommand{\footrulewidth}{0pt}}
\pagestyle{fancy}
\fancyhf{}
\fancyhead[LO]{MLG -- 2018}
\fancyhead[RO,LE]{\thepage}
\renewcommand{\headrulewidth}{0.4pt}
\renewcommand{\footrulewidth}{0pt}
%
\usepackage{sectsty}
\allsectionsfont{\color{DarkGreen}}
\title{\color{DarkGreen}\huge\bfseries Data Management : Lab 1}
\author{Antoine Friant}
\date{\today}

\begin{document}
	
	\maketitle
	%rendu : lab1\_friant\_ransome.zip	
	%contient : report.pdf + sources ("src/" + "pom.xml")
	
	\section{Understanding the Lucene API}
	
	\subsection{Does the command line demo use stopword removal?}
	Yes, because the query "a an or and at query response" returned the exact same results as "query response". If it weren't using stopword removal, we'd get less results when using stopwords.
	
	\subsection{"Does the command line demo use stemming ?"}
	No, because the queries "query" and "queries" don't give the same number of results.
	
	\subsection{"Is the search of the command line demo case insensitive ?"}
	Yes, because "QUERY" and "query" give the same results.
	
	\subsection{"Does it matter whether stemming occurs before or after stopword removal ?"}
	It only matters if stopwords find themselves grouped with non stopwords after the stemming step. If such a thing is possible, then we need to remove stopwords before stemming. If we don't, words from the query could be ignored or stopwords could change the result of the query.
	
	\section{Indexing and Searching the CACM collection}
	\subsection{Explain which field type can be used for id, title, summary and author}
	We need an integer for "id" so LongPoint or IntPoint. The others need to be TextField to perform full-text searches. We chose LongPoint since the id is type long. In order to store it, we also create an StoreField "id".
	
	\subsection{Using different Analyzers}
	\subsubsection{StandardAnalyzer}
	\begin{itemize}
		\item Indexed documents : 3'203
		\item Indexed terms : 26'634
		\item Indexed terms in the summary field : 19'972
		\item The top 10 frequent terms of the summary field in the index : which, system, computer, paper, can, described, given, presented, time, from
		\item Size of the index on disk : 2.2 MB
		\item Required time for indexing : 965 ms
	\end{itemize}

	\subsubsection{WhitespaceAnalyzer}
\begin{itemize}
	\item Indexed documents : 3'203
	\item Indexed terms : 34'443
	\item Indexed terms in the summary field : 26'821
	\item The top 10 frequent terms of the summary field in the index : of, the, is, a, and, to, in, for, The, are
	\item Size of the index on disk : 2.6 MB
	\item Required time for indexing : 971 ms
\end{itemize}

	\subsubsection{EnglishAnalyzer}
\begin{itemize}
	\item Indexed documents : 3'203
	\item Indexed terms : 22'513
	\item Indexed terms in the summary field : 16'724
	\item The top 10 frequent terms of the summary field in the index : us, which, comput, program, system, present, describ, paper, method, can
	\item Size of the index on disk : 2.1 MB
	\item Required time for indexing : 1'129 ms
\end{itemize}

	\subsubsection{ShingleAnalyzerWrapper (shingle size 2)}
\begin{itemize}
	\item Indexed documents : 3'203
	\item Indexed terms : 105'802
	\item Indexed terms in the summary field : 85'610
	\item The top 10 frequent terms of the summary field in the index : which, system, paper, computer, can, paper, described, given, presented, time
	\item Size of the index on disk : 4.8 MB
	\item Required time for indexing : 1'682 ms
\end{itemize}

	\subsubsection{ShingleAnalyzerWrapper (shingle size 3)}
\begin{itemize}
	\item Indexed documents : 3'203
	\item Indexed terms : 147'205
	\item Indexed terms in the summary field : 125'776
	\item The top 10 frequent terms of the summary field in the index : which, system, computer, paper, can, described, time, given, presented, from
	\item Size of the index on disk : 6.3 MB
	\item Required time for indexing : 1'751 ms
\end{itemize}

	\subsubsection{StopAnalyzer}
\begin{itemize}
	\item Indexed documents : 3'203
	\item Indexed terms : 24'507
	\item Indexed terms in the summary field : 18'658
	\item The top 10 frequent terms of the summary field in the index : which, system, computer, paper, described, can, presented, given, time, from
	\item Size of the index on disk : 2.2 MB
	\item Required time for indexing : 812 ms
\end{itemize}
	
	\subsection{Reading Index}
	\texttt{Most frequent terms in authors :}\\
	\texttt{38 : Thacher Jr., H. C.}

\texttt{Most frequent terms in title :}\\
\texttt{983 : algorithm}\\
\texttt{262 : computer}\\
\texttt{172 : system}\\
\texttt{154 : programming}\\
\texttt{124 : method}\\
\texttt{112 : data}\\
\texttt{109 : systems}\\
\texttt{101 : language}\\
\texttt{93 : program}\\
\texttt{84 : time}
	\begin{lstlisting}
// file Main.java
private static Analyzer getAnalyzer() {	
	// treat the author field as a single word
	Map<String, Analyzer> analyzerMap = new HashMap<>();
	analyzerMap.put("authors", new KeywordAnalyzer());
	return new PerFieldAnalyzerWrapper(
	new StandardAnalyzer(), analyzerMap);
}

// file QueriesPerformer.java
public void printTopRankingTerms(String field, int numTerms) {
	// This methods print the top ranking term for a field.
	try {
		TermStats[] foundTerms = HighFreqTerms.getHighFreqTerms(indexReader, numTerms, field,
		new HighFreqTerms.TotalTermFreqComparator());
		
		System.out.println("\nMost frequent terms in " + field + " :");
		
		for (TermStats term : foundTerms) {
			System.out.println(term.totalTermFreq + " : " + term.termtext.utf8ToString());
		}
	} catch (Exception e) {
		e.printStackTrace();
	}
}
	\end{lstlisting}
	
	\subsection{Searching}
	
	\begin{lstlisting}
// file Main.java
private static void searching(QueriesPerformer queriesPerformer) {
	queriesPerformer.query("\"Information Retrieval\"");
	queriesPerformer.query("Information Retrieval");
	queriesPerformer.query("Information +Retrieval -Database");
	queriesPerformer.query("Info*");
	queriesPerformer.query("\"Information Retrieval\"~5");
}

// file QueriesPerformer.java
public void query(String q) {
	final int LIMIT = 10;
	
	QueryParser parser = new QueryParser("summary", analyzer);
	
	try {
		Query query = parser.parse(q);
		IndexSearcher searcher = new IndexSearcher(indexReader);
		searcher.search(query, 100);
		
		TopDocs results = searcher.search(query, LIMIT);
		ScoreDoc[] hits = results.scoreDocs;
		
		System.out.println("\nSearching for: " + q + " (" + results.totalHits + " results)");
		
		for (int i = 0; i < LIMIT && i < results.totalHits; ++i) {
			Document doc = searcher.doc(hits[i].doc);
			System.out.println(doc.get("id") + ": " + doc.get("title"));
		}
		
		System.out.println();
	} catch (Exception e) {
		e.printStackTrace();
	}

System.out.println("\n");
}

// Output :
Searching for: "Information Retrieval" (11 results)
891: Everyman's Information Retrieval System
1457: Data Manipulation and Programming Problemsin Automatic Information Retrieval
1935: Randomized Binary Search Technique
1699: Experimental Evaluation of InformationRetrieval Through a Teletypewriter
2516: Hierarchical Storage in Information Retrieval
2519: On the Problem of Communicating Complex Information
2307: Dynamic Document Processing
2795: Sentence Paraphrasing from a Conceptual Base
2990: Effective Information Retrieval Using Term Accuracy
1652: A Code for Non-numeric Information ProcessingApplications in Online Systems

Searching for: Information Retrieval (188 results)
1457: Data Manipulation and Programming Problemsin Automatic Information Retrieval
891: Everyman's Information Retrieval System
1699: Experimental Evaluation of InformationRetrieval Through a Teletypewriter
2307: Dynamic Document Processing
3134: The Use of Normal Multiplication Tablesfor Information Storage and Retrieval
1032: Theoretical Considerations in Information Retrieval Systems
1935: Randomized Binary Search Technique
1681: Easy English,a Language for InformationRetrieval Through a Remote Typewriter Console
2990: Effective Information Retrieval Using Term Accuracy
2519: On the Problem of Communicating Complex Information

Searching for: Information +Retrieval -Database (54 results)
1457: Data Manipulation and Programming Problemsin Automatic Information Retrieval
891: Everyman's Information Retrieval System
1699: Experimental Evaluation of InformationRetrieval Through a Teletypewriter
2307: Dynamic Document Processing
3134: The Use of Normal Multiplication Tablesfor Information Storage and Retrieval
1032: Theoretical Considerations in Information Retrieval Systems
1935: Randomized Binary Search Technique
1681: Easy English,a Language for InformationRetrieval Through a Remote Typewriter Console
2990: Effective Information Retrieval Using Term Accuracy
2519: On the Problem of Communicating Complex Information

Searching for: Info* (193 results)
222: Coding Isomorphisms
272: A Storage Allocation Scheme for ALGOL 60
396: Automation of Program  Debugging
397: A Card Format for Reference Files in Information Processing
409: CL-1, An Environment for a Compiler
440: Record Linkage
483: On the Nonexistence of a Phrase Structure Grammar for ALGOL 60
616: An Information Algebra - Phase I Report-LanguageStructure Group of the CODASYL Development Committee
644: A String Language for Symbol Manipulation Based on ALGOL 60
655: COMIT as an IR Language

Searching for: "Information Retrieval"~5 (15 results)
891: Everyman's Information Retrieval System
1457: Data Manipulation and Programming Problemsin Automatic Information Retrieval
1935: Randomized Binary Search Technique
1699: Experimental Evaluation of InformationRetrieval Through a Teletypewriter
2307: Dynamic Document Processing
2516: Hierarchical Storage in Information Retrieval
2519: On the Problem of Communicating Complex Information
2795: Sentence Paraphrasing from a Conceptual Base
2990: Effective Information Retrieval Using Term Accuracy
1652: A Code for Non-numeric Information ProcessingApplications in Online Systems

	\end{lstlisting}
	
	\subsection{Tuning the Lucene Score}
	
\begin{lstlisting}
// MySimilarity.java
package ch.heigvd.iict.dmg.labo1.similarities;

import org.apache.lucene.index.FieldInvertState;
import org.apache.lucene.search.similarities.ClassicSimilarity;

public class MySimilarity extends ClassicSimilarity {
	@Override
	public float tf(float freq) {
		return 1 + (float)Math.log(freq);
	}
	
	@Override
	public float idf(long docFreq, long numDocs) {
		return (float)Math.log(numDocs/((float)docFreq + 1)) + 1;
	}
	
	@Override
	public float coord(int overlap, int maxOverlap) {
		return (float)Math.sqrt(overlap/(float)maxOverlap);
	}
	
	@Override
	public float lengthNorm(FieldInvertState state) {
		return 1;
	}
}

// Output :
Searching for: Information Retrieval (188 results)
1032: Theoretical Considerations in Information Retrieval Systems
1457: Data Manipulation and Programming Problemsin Automatic Information Retrieval
3134: The Use of Normal Multiplication Tablesfor Information Storage and Retrieval
891: Everyman's Information Retrieval System
1699: Experimental Evaluation of InformationRetrieval Through a Teletypewriter
2307: Dynamic Document Processing
1527: A Grammar Base Question Answering Procedure
1681: Easy English,a Language for InformationRetrieval Through a Remote Typewriter Console
2990: Effective Information Retrieval Using Term Accuracy
1652: A Code for Non-numeric Information ProcessingApplications in Online Systems
\end{lstlisting}	


\begin{table}[]
	\begin{center}
	\begin{tabular}{r|l}
	Lucene classic scoring	& TF-IDF scoring \\\hline
	1457 & 1032\\
	891 & 1457\\
	1699 & 3134\\
	2307 & 891\\
	3134 & 1699\\
	1032 & 2307\\
	1935 & 1527\\
	1681 & 1681\\
	2990 & 2990\\
	2519 & 1652
	\end{tabular}
	\caption{Comparison of the top 10 results using ClassicSimilarity and MySimilarity (document ids). The number of results are the same, but their ranks are different. Only two documents disappear from the top 10 results, so the change is not dramatic overall.}
	\end{center}
\end{table}
\end{document}
 
